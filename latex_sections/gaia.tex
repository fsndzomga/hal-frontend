\subsection{GAIA}\label{app:gaia}

\paragraph{Benchmark.}
GAIA is a benchmark for General AI Assistants that requires a set of fundamental abilities such as reasoning, multi-modality handling, web browsing, and tool-use proficiency. It contains 450 questions with unambiguous answers, requiring different levels of tooling and autonomy to solve. It is divided in 3 levels, where level 1 should be breakable by very good LLMs, and level 3 indicate a strong jump in model capabilities. We evaluate on the public validation set of 165 questions.
Paper: \cite{gaia}.

\paragraph{Agents.}
Briefly describe the two agents you ran (scaffolds, models, reasoning settings).


\begin{table}[t]
  \centering
  \caption{GAIA Leaderboard (verbatim from the website).}
  \label{tab:gaia_full}
  \begin{tabular}{lcccc}
\toprule
Scaffold & Model & Accuracy & Cost (USD) & Pareto Optimal \\
\midrule
HAL Generalist Agent & Claude Opus 4 High (May 2025) & 64.8\% & \$665.89 & Yes \\
HAL Generalist Agent & Claude-3.7 Sonnet High (February 2025) & 64.2\% & \$122.49 & Yes \\
HF Open Deep Research & GPT-5 Medium (August 2025) & 62.8\% & \$359.83 &  \\
HAL Generalist Agent & o4-mini Low (April 2025) & 58.2\% & \$73.26 & Yes \\
HF Open Deep Research & Claude Opus 4 (May 2025) & 57.6\% & \$1686.07 &  \\
HAL Generalist Agent & Claude-3.7 Sonnet (February 2025) & 56.4\% & \$130.68 &  \\
HF Open Deep Research & o4-mini High (April 2025) & 55.8\% & \$184.87 &  \\
HAL Generalist Agent & o4-mini High (April 2025) & 54.5\% & \$59.39 & Yes \\
HF Open Deep Research & GPT-4.1 (April 2025) & 50.3\% & \$109.88 &  \\
HAL Generalist Agent & GPT-4.1 (April 2025) & 49.7\% & \$74.19 &  \\
HF Open Deep Research & o4-mini Low (April 2025) & 47.9\% & \$80.80 &  \\
HF Open Deep Research & Claude-3.7 Sonnet (February 2025) & 37.0\% & \$415.15 &  \\
HAL Generalist Agent & DeepSeek V3 (March 2025) & 36.4\% & \$29.27 &  \\
HF Open Deep Research & Claude-3.7 Sonnet High (February 2025) & 35.8\% & \$113.65 &  \\
HAL Generalist Agent & Gemini 2.0 Flash (February 2025) & 32.7\% & \$7.80 & Yes \\
HF Open Deep Research & o3 Medium (April 2025) & 32.7\% & \$136.39 &  \\
HAL Generalist Agent & DeepSeek R1 (January 2025) & 30.3\% & \$14.92 &  \\
HAL Generalist Agent & Claude Opus 4 (May 2025) & 30.3\% & \$272.76 &  \\
HF Open Deep Research & DeepSeek V3 (March 2025) & 28.5\% & \$76.64 &  \\
HF Open Deep Research & Claude Opus 4.1 (August 2025) & 28.5\% & \$1306.85 &  \\
HF Open Deep Research & Claude Opus 4.1 High (August 2025) & 25.4\% & \$1473.64 &  \\
HF Open Deep Research & DeepSeek R1 (January 2025) & 24.9\% & \$30.47 &  \\
HF Open Deep Research & Gemini 2.0 Flash (February 2025) & 19.4\% & \$18.82 &  \\
\bottomrule
\end{tabular}
\end{table}


\begin{figure}[htbp]
  \centering
  \includegraphics[width=0.9\linewidth]{gaia_pareto_accuracy_vs_cost.pdf}
  \caption{Pareto frontier of accuracy vs.\ cost.}
  \label{fig:gaia_pareto}
\end{figure}

\begin{figure}[htbp]
  \centering
  \includegraphics[width=0.9\linewidth]{gaia_total_tokens.pdf}
  \caption{Total completion tokens used per Agent}
  \label{fig:gaia_tokens}
\end{figure}

\begin{figure*}[t]
  \centering
  \adjustbox{max width=\textwidth, max height=0.9\textheight}{%
    \includegraphics{gaia_heatmap_best_vs_any.pdf}%
  }
  \caption{Heatmap: best-agent vs.\ any-agent success.}
  \label{fig:gaia_heatmap}
\end{figure*}

\begin{figure}[htbp]
  \centering
  \includegraphics[width=0.9\linewidth]{gaia_accuracy_vs_release_date.pdf}
  \caption{Accuracy vs.\ model release date.}
  \label{fig:gaia_release}
\end{figure}

\clearpage