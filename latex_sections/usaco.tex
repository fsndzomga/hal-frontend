\subsection{USACO}\label{app:usaco}

\paragraph{Benchmark.}
The USACO benchmark evaluates AI agents on competitive programming problems from the USA Computing Olympiad. It consists of 307 problems, complete with exhaustive test cases, problem analyses, and difficulty labels. This benchmark tests agents' ability to solve algorithmic challenges and write efficient code.
Paper: \cite{usaco}.

\paragraph{Agents.}
Briefly describe the two agents you ran (scaffolds, models, reasoning settings).


\begin{table}[t]
  \centering
  \caption{USACO Leaderboard (verbatim from the website).}
  \label{tab:usaco_full}
  \begin{tabular}{lcccc}
\toprule
Scaffold & Model & Accuracy & Cost (USD) & Pareto Optimal \\
\midrule
USACO Episodic + Semantic & GPT-5 Medium (August 2025) & 69.7\% & \$64.13 & Yes \\
USACO Episodic + Semantic & o4-mini High (April 2025) & 58.0\% & \$44.04 & Yes \\
USACO Episodic + Semantic & Claude Opus 4.1 High (August 2025) & 51.5\% & \$267.72 &  \\
USACO Episodic + Semantic & Claude Opus 4.1 (August 2025) & 48.2\% & \$276.19 &  \\
USACO Episodic + Semantic & o3 Medium (April 2025) & 46.2\% & \$57.30 &  \\
USACO Episodic + Semantic & GPT-4.1 (April 2025) & 45.0\% & \$28.10 &  \\
USACO Episodic + Semantic & DeepSeek V3 (March 2025) & 39.1\% & \$12.08 & Yes \\
USACO Episodic + Semantic & DeepSeek R1 (January 2025) & 38.1\% & \$22.40 &  \\
USACO Episodic + Semantic & o4-mini Low (April 2025) & 30.9\% & \$21.14 &  \\
USACO Episodic + Semantic & Claude-3.7 Sonnet (February 2025) & 29.3\% & \$38.70 &  \\
USACO Episodic + Semantic & Gemini 2.0 Flash (February 2025) & 27.0\% & \$1.46 & Yes \\
USACO Episodic + Semantic & Claude-3.7 Sonnet High (February 2025) & 26.7\% & \$56.43 &  \\
HAL Generalist Agent & GPT-4.1 (April 2025) & 25.4\% & \$197.33 &  \\
\bottomrule
\end{tabular}
\end{table}


\begin{figure}[htbp]
  \centering
  \includegraphics[width=0.9\linewidth]{usaco_pareto_accuracy_vs_cost.pdf}
  \caption{Pareto frontier of accuracy vs.\ cost.}
  \label{fig:usaco_pareto}
\end{figure}

\begin{figure}[htbp]
  \centering
  \includegraphics[width=0.9\linewidth]{usaco_total_tokens.pdf}
  \caption{Total completion tokens used per Agent}
  \label{fig:usaco_tokens}
\end{figure}

\begin{figure*}[t]
  \centering
  \adjustbox{max width=\textwidth, max height=0.9\textheight}{%
    \includegraphics{usaco_heatmap_best_vs_any.pdf}%
  }
  \caption{Heatmap: best-agent vs.\ any-agent success.}
  \label{fig:usaco_heatmap}
\end{figure*}

\begin{figure}[htbp]
  \centering
  \includegraphics[width=0.9\linewidth]{usaco_accuracy_vs_release_date.pdf}
  \caption{Accuracy vs.\ model release date.}
  \label{fig:usaco_release}
\end{figure}

\clearpage