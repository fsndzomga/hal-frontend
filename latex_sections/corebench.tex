\subsection{CORE-Bench}\label{app:corebench}

\paragraph{Benchmark.}
CORE-Bench evaluates the ability of agents to computationally reproduce the results of published scientific papers. In CORE-Bench Medium, the agent is given a Dockerfile and instructions on how to use the Dockerfile to fully reproduce the paper. This level mainly evaluates agents ability to use and interact with the terminal. The agent must then answer questions about the output of the code.
Paper: \cite{corebench}.

\paragraph{Agents.}
Briefly describe the two agents you ran (scaffolds, models, reasoning settings).


\begin{table}[t]
  \centering
  \caption{CORE-Bench Leaderboard (verbatim from the website).}
  \label{tab:corebench_full}
  \begin{tabular}{lcccc}
\toprule
Scaffold & Model & Accuracy & Cost (USD) & Pareto Optimal \\
\midrule
CORE-Agent & Claude Opus 4.1 (August 2025) & 51.1\% & \$412.42 & Yes \\
CORE-Agent & Claude Opus 4.1 High (August 2025) & 42.2\% & \$509.95 &  \\
HAL Generalist Agent & Claude-3.7 Sonnet High (February 2025) & 37.8\% & \$66.15 & Yes \\
HAL Generalist Agent & o4-mini High (April 2025) & 35.6\% & \$45.37 & Yes \\
CORE-Agent & Claude-3.7 Sonnet (February 2025) & 35.6\% & \$73.04 &  \\
HAL Generalist Agent & Claude Opus 4.1 (August 2025) & 35.6\% & \$375.11 &  \\
CORE-Agent & Claude Sonnet 4 High (May 2025) & 33.3\% & \$100.48 &  \\
CORE-Agent & GPT-4.1 (April 2025) & 33.3\% & \$107.36 &  \\
HAL Generalist Agent & Claude Opus 4.1 High (August 2025) & 33.3\% & \$358.47 &  \\
HAL Generalist Agent & Claude-3.7 Sonnet (February 2025) & 31.1\% & \$56.64 &  \\
CORE-Agent & Claude Sonnet 4 (May 2025) & 28.9\% & \$50.27 &  \\
CORE-Agent & GPT-5 Medium (August 2025) & 26.7\% & \$31.76 &  \\
CORE-Agent & o4-mini High (April 2025) & 26.7\% & \$61.35 &  \\
CORE-Agent & Claude-3.7 Sonnet High (February 2025) & 24.4\% & \$72.47 &  \\
CORE-Agent & o3 Medium (April 2025) & 24.4\% & \$120.47 &  \\
HAL Generalist Agent & GPT-4.1 (April 2025) & 22.2\% & \$58.32 &  \\
HAL Generalist Agent & o3 Medium (April 2025) & 22.2\% & \$88.34 &  \\
CORE-Agent & Gemini 2.5 Pro Preview (March 2025) & 22.2\% & \$182.34 &  \\
CORE-Agent & DeepSeek V3.1 (August 2025) & 20.0\% & \$12.55 & Yes \\
CORE-Agent & DeepSeek V3 (March 2025) & 17.8\% & \$25.26 &  \\
CORE-Agent & o4-mini Low (April 2025) & 17.8\% & \$31.79 &  \\
HAL Generalist Agent & o4-mini Low (April 2025) & 15.6\% & \$22.50 &  \\
CORE-Agent & GPT-OSS-120B (August 2025) & 11.1\% & \$4.21 &  \\
CORE-Agent & GPT-OSS-120B High (August 2025) & 11.1\% & \$4.21 &  \\
CORE-Agent & Gemini 2.0 Flash (February 2025) & 11.1\% & \$12.46 &  \\
HAL Generalist Agent & GPT-5 Medium (August 2025) & 11.1\% & \$29.75 &  \\
HAL Generalist Agent & GPT-OSS-120B High (August 2025) & 8.9\% & \$2.05 & Yes \\
HAL Generalist Agent & GPT-OSS-120B (August 2025) & 8.9\% & \$2.79 &  \\
HAL Generalist Agent & DeepSeek V3 (March 2025) & 8.9\% & \$4.69 &  \\
HAL Generalist Agent & DeepSeek R1 (May 2025) & 8.9\% & \$7.77 &  \\
CORE-Agent & DeepSeek R1 (January 2025) & 6.7\% & \$15.95 &  \\
HAL Generalist Agent & Gemini 2.0 Flash (February 2025) & 4.4\% & \$7.06 &  \\
HAL Generalist Agent & DeepSeek R1 (January 2025) & 2.2\% & \$3.00 &  \\
\bottomrule
\end{tabular}
\end{table}


\begin{figure}[htbp]
  \centering
  \includegraphics[width=0.9\linewidth]{core_pareto_accuracy_vs_cost.pdf}
  \caption{Pareto frontier of accuracy vs.\ cost.}
  \label{fig:core_pareto}
\end{figure}

\begin{figure}[htbp]
  \centering
  \includegraphics[width=0.9\linewidth]{core_total_tokens.pdf}
  \caption{Total completion tokens used per Agent}
  \label{fig:core_tokens}
\end{figure}

\begin{figure*}[t]
  \centering
  \adjustbox{max width=\textwidth, max height=0.9\textheight}{%
    \includegraphics{core_heatmap_best_vs_any.pdf}%
  }
  \caption{Heatmap: best-agent vs.\ any-agent success.}
  \label{fig:core_heatmap}
\end{figure*}

\begin{figure}[htbp]
  \centering
  \includegraphics[width=0.9\linewidth]{core_accuracy_vs_release_date.pdf}
  \caption{Accuracy vs.\ model release date.}
  \label{fig:core_release}
\end{figure}

\clearpage