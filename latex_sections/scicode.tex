\subsection{SciCode}\label{app:scicode}

\paragraph{Benchmark.}
SciCode evaluates AI agents' ability to generate code for realistic scientific research tasks. It is made up of 65 main problems decomposed into 338 subproblems across 16 subfields in six natural science domains (Mathematics, Physics, Chemistry, Biology, Material Science, and Computational Mechanics).
Paper: \cite{scicode}.

\paragraph{Agents.}
Briefly describe the two agents you ran (scaffolds, models, reasoning settings).


\begin{table}[t]
  \centering
  \caption{SciCode Leaderboard (verbatim from the website).}
  \label{tab:scicode_full}
  \begin{tabular}{lcccc}
\toprule
Scaffold & Model & Accuracy & Cost (USD) & Pareto Optimal \\
\midrule
Scicode Zero Shot Agent & o4-mini Low (April 2025) & 9.2\% & \$1.74 & Yes \\
Scicode Tool Calling Agent & o3 Medium (April 2025) & 9.2\% & \$111.11 &  \\
Scicode Tool Calling Agent & Claude Opus 4.1 (August 2025) & 7.7\% & \$625.13 &  \\
Scicode Tool Calling Agent & Claude Opus 4.1 High (August 2025) & 6.9\% & \$550.54 &  \\
Scicode Zero Shot Agent & GPT-4.1 (April 2025) & 6.2\% & \$2.82 &  \\
Scicode Zero Shot Agent & o4-mini High (April 2025) & 6.2\% & \$5.37 &  \\
Scicode Tool Calling Agent & GPT-5 Medium (August 2025) & 6.2\% & \$193.52 &  \\
Scicode Zero Shot Agent & o3 Medium (April 2025) & 4.6\% & \$6.03 &  \\
Scicode Tool Calling Agent & o4-mini Low (April 2025) & 4.6\% & \$46.30 &  \\
Scicode Tool Calling Agent & o4-mini High (April 2025) & 4.6\% & \$66.20 &  \\
Scicode Tool Calling Agent & Claude-3.7 Sonnet High (February 2025) & 4.6\% & \$204.37 &  \\
Scicode Zero Shot Agent & DeepSeek V3 (March 2025) & 3.1\% & \$0.79 &  \\
Scicode Zero Shot Agent & Claude-3.7 Sonnet High (February 2025) & 3.1\% & \$4.99 &  \\
Scicode Tool Calling Agent & Claude-3.7 Sonnet (February 2025) & 3.1\% & \$191.41 &  \\
Scicode Zero Shot Agent & Gemini 2.0 Flash (February 2025) & 1.5\% & \$0.12 & Yes \\
Scicode Tool Calling Agent & Gemini 2.0 Flash (February 2025) & 1.5\% & \$5.23 &  \\
Scicode Tool Calling Agent & GPT-4.1 (April 2025) & 1.5\% & \$69.39 &  \\
Scicode Zero Shot Agent & DeepSeek R1 (May 2025) & 0.0\% & \$2.19 &  \\
Scicode Zero Shot Agent & Claude-3.7 Sonnet (February 2025) & 0.0\% & \$5.10 &  \\
Scicode Tool Calling Agent & DeepSeek V3 (March 2025) & 0.0\% & \$52.11 &  \\
Scicode Tool Calling Agent & DeepSeek R1 (May 2025) & 0.0\% & \$57.62 &  \\
\bottomrule
\end{tabular}
\end{table}


\begin{figure}[htbp]
  \centering
  \includegraphics[width=0.9\linewidth]{scicode_pareto_accuracy_vs_cost.pdf}
  \caption{Pareto frontier of accuracy vs.\ cost.}
  \label{fig:scicode_pareto}
\end{figure}

\begin{figure}[htbp]
  \centering
  \includegraphics[width=0.9\linewidth]{scicode_total_tokens.pdf}
  \caption{Total completion tokens used per Agent}
  \label{fig:scicode_tokens}
\end{figure}

\begin{figure*}[t]
  \centering
  \adjustbox{max width=\textwidth, max height=0.9\textheight}{%
    \includegraphics{scicode_heatmap_best_vs_any.pdf}%
  }
  \caption{Heatmap: best-agent vs.\ any-agent success.}
  \label{fig:scicode_heatmap}
\end{figure*}

\begin{figure}[htbp]
  \centering
  \includegraphics[width=0.9\linewidth]{scicode_accuracy_vs_release_date.pdf}
  \caption{Accuracy vs.\ model release date.}
  \label{fig:scicode_release}
\end{figure}

\clearpage